
Time series data has long been central in machine learning, data mining, and scientific computing in general. Thanks to recent exponential growth in the number of connected, mobile, and wearable devices, it is gaining even more importance, with billions of sensors now streaming data about quantities ranging from acceleration to temperature to power consumption.

In order to reduce transmission, power, and storage costs, it is highly desirable to compress this data. For certain types of time series, such as audio \cite{flac, ...} and timestamps \cite{someLemire}, excellent methods already exist. For other time series, this is not the case. One can apply generic compression algorithms, such as GZIP \cite{gzip}, ZSTD \cite{zstd}, LZ4 \cite{lz4}, etc, but as we show, these methods tend to perform poorly on time series. One can also apply integer compression algorithms, such as FastPFOR \cite{fastpfor}, but these suffer from the same drawback.

Moreover, almost all existing methods assume computational resources that may be unavailable on low-power devices collecting and transmitting data. In particular, such devices may have less than 1KB of writable memory.

The contribution of this work is the introduction of \mine (Sprintz PReserves INteger Time SerieZ), a lossless compression algorithm for time series that does not assume a particular application domain (e.g., music) and is suitable for execution on low-power hardware. As we show experimentally, \mine achieves higher compression ratios than any other method across a wide range of datasets while maintaining decompression speeds over \%d MB/s in a single thread.

% ------------------------------------------------
\subsection{Why lossless?}
% ------------------------------------------------

% Given that time series are often represented with too high a sampling rate and bit depth relative to the level of noise

% and often oversampled or quantized with unneeded precision

Given that time series are almost always noisy and often oversampled, it might seem unnecessary to compress them losslessly. %---i.e., if the details lost are mostly noise, why bother preserving them?

However, note that noise and oversampling tend to 1) vary across applications, and 2) be possible to address in an application-specific way as a preprocessing step. Consequently, instead of coupling some particular smoothing, downsampling, or distortion to the compression algorithm and assuming that the result will be ``good enough'' for all data, it is better for the compression algorithm to preserve what it is given and leave preprocessing up to the developer.

% assuming that some level of downsampling or some particular smoothing will be ``good enough'' for all data, it is better for the compression algorithm to preserve what it is given and leave preprocessing up to the developer.

% the nature of the noise, oversampling, etc, varies across applications, and 2) it is easy to preprocess the data in an application-specific way to address the before compressing it. Consequently, instead of assuming that some level of downsampling or some particular smoothing will be ``good enough'' for all data, it is better for the compression algorithm to preserve what it is given and leave preprocessing up to the developer.

% Consequently, it is undesirable to couple denoising to the compression algorithm. Instead, engineers should be free to filter, downsample, quantize, and otherwise condition the data in any way they see fit, and trust that the compression algorithm will preserve the end result. % This applies equally to other operations such as downsampling and reduction of bit depth.

Moreover, companies cannot necessarily anticipate what sort of degradation might reduce their data's utility for future analysis; this makes haphazardly degrading it through lossy compression risky.

% ------------------------------------------------
\subsection{Limited hardware}
% ------------------------------------------------

Many connected devices are powered by batteries or harvested energy \cite{calhoun}. This results in strict power budgets and, in order to satisfy them, omission of certain functionality. In particular, many devices lack hardware support for floating point operations, SIMD (vector) instructions, and integer division. Moreover, they often have no more than a few KB of memory, clocks in the tens of MHz at most, and 8-, 16-, or 32-bit processors instead of 64-bit \cite{cc2540, intelWhatever}.

We do not assume that the hardware decompressing the data is so limited. Instead, this hardware is likely a modern x86 server with SIMD instructions, gigabytes of RAM, and a multi-GHz clock.

Given these differing capabilities, a natural question is whether the data must be compressed at the device instead of at the server. Unfortunately, delaying the compression would eliminate much of its benefit. The reason is that reducing the amount of data the device must send offers enormous power savings. To a first approximation, sending data over Bluetooth Low Energy (BLE) costs \%d mW while computing at full power costs \%d mW and sitting idle costs only \%d mW \cite{TODO}. Furthermore, compressing the data before the server reduces network load.

% As a consequence of limited power budgets, there is a strong need to reduce the amount of data transmitted or stored. Indeed, this is the driving motivation for our work. To a first approximation, sending data over Bluetooth Low Energy costs \%d mW and writing to an SD card costs \%d mW; in contrast, computing at full power costs \%d mW and sitting idle costs only \%d mW. This implies that it is worth a great deal of computation to reduce the sizes of writes and transmits.

% ------------------------------------------------
\subsection{Nature of the data}
% ------------------------------------------------

Almost any time series reflecting a real-world signal will be digitized using an analog-to-digital converter (ADC). This means that the data will be represented as integers of at most 24 bits \cite{digikeySearch}, and typically 16 or fewer. For example, even lossless audio codecs store only 16 bits \cite{someAudioCodecs, TODO}. Furthermore, there is empirical evidence that most time series can be quantized to 8 or fewer bits with little or no loss of information (as measured by misclassification rate) \cite{ucrWhateverItWas}. Consequently, we focus on compressing 8b and 16b integer time series. % every ADC currently available through DigiKey, for example, supports 24 or fewer bits as of this writing \cite{digikeyADC}; even high-quality audio is only 16 bits \cite{someAudioCodec}.

Our method could also be applied to floating point time series insofar as they could be quantized to integers, but we do not evaluate this experimentally since it depends on the quantization method.

In addition to being stored as integers, the data is likely to have two additional characteristics. Specifically, it is likely to be 1) constant for long stretches of time and 2) always or sometimes quasi-periodic with unknown period(s). A temperature sensor, for example, will observe nearly identical temperatures for many seconds or minutes, while also experiencing predictable variations over the course of both each day and each year. Similarly, a smart watch will observe long periods of limited movement during sleep or computer use but repetitive movement during walking, running, or swimming. Our method does not require these characteristics to be present, but is designed to exploit them when they are.

Finally, we assume that the time series to be compressed are univariate--i.e., they are sequences of scalars. One might be able to obtain better compression by jointly compressing multiple variables known \textit{a priori} to be correlated (e.g. X-axis and Y-axis acceleration), but we do not do this since:

\begin{enumerate}[leftmargin=4mm]
\item It results in read amplification if only one variable is of interest.
\item It tends not to increase compression very much.
\item It complicates the algorithm, since inter-variable correlations can switch unpredictably between being positive and negative.
\item It makes apples-to-apples comparison to existing work more difficult.
\end{enumerate}

% ------------------------------------------------
% \subsection{Limited hardware}
% ------------------------------------------------



