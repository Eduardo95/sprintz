
% \documentclass[conference]{IEEEtran}
% \documentclass{sig-alternate} % pre 2017
% \documentclass[sigconf]{acmart}  % starting in 2017
\documentclass{vldb}
\usepackage{balance}  % vldb requirement; for  \balance command on last page
%-------------------------------------------------------------- Includes

\usepackage{bbm}  % who knows?

\usepackage{amsmath}          % basic math
\usepackage{amssymb} 			    % math symbols
% \usepackage{amsthm}           % theorems
\usepackage{textcomp}

% \usepackage{float}            % make figures work
% \usepackage{cite}             % citations would be nice
\usepackage{url}              % better urls; magically keeping doc from breaking due to urls in refs

% \usepackage{tabu}
\usepackage{array}            % multiline table cells; somehow
\usepackage{tabularx}         % better tables
% \usepackage[table]{xcolor}    % colored table cells  % incompatible with sigconf
\usepackage{colortbl}
\newcolumntype{Y}{>{\centering\arraybackslash}X}	% centered column type for tabularx

% stuff for piecewise functions
\usepackage{mathtools}          %loads amsmath as well
\DeclarePairedDelimiter\Floor\lfloor\rfloor
\DeclarePairedDelimiter\Ceil\lceil\rceil

\DeclareMathOperator*{\argmin}{arg\,min} % argmin
\DeclareMathOperator*{\argmax}{arg\,max} % argmax

\usepackage{pbox}   % for trick to force linebreaks in table cells

\usepackage{setspace}
% \usepackage{setspace}

% \usepackage{booktabs} % acm recommended, and for pandas latex tables

%-------------------------------------------------------------- Algorithm setup

% \usepackage{algorithm2e}
\usepackage{algorithm}
\usepackage[noend]{algpseudocode} % I think this removes trailing "end {if,for,while}"
\usepackage{algorithmicx}

% \algnewcommand{\LineComment}[1]{\State \(\triangleright\) #1} % left-aligned comments
% \algnewcommand{\SideComment}[1]{\(//\) #1}
\algnewcommand{\COMMENT}[2][.5\linewidth]{\leavevmode\hfill\makebox[#1][l]{//~#2}}
\algnewcommand{\COMMENTT}[2][.5\linewidth]{\leavevmode\hfill\makebox[#1][l]{\hphantom{aa} // ~#2}}
\algnewcommand{\LineComment}[1]{\State \(//\) #1}	% left-aligned comments
\algnewcommand\RETURN{\State \textbf{return} }

\newenvironment{struct}[1][htb]
  {\renewcommand{\algorithmcfname}{Class}% Update algorithm name
   \begin{algorithm}[#1]%
  }{\end{algorithm}}

%-------------------------------------------------------------- Figures setup

% \usepackage[pdftex]{graphicx}
\usepackage{graphicx}
\usepackage[space]{grffile}   % allow spaces in file names
% declare the path(s) where your graphic files are
% \graphicspath{{../figs/}}
\graphicspath{{figs/}}
% and their extensions so you won't have to specify these
\DeclareGraphicsExtensions{.pdf,.jpeg,.jpg,.png}


%\textfloatsep: space between last top float or first bottom float and the text (default = 20.0pt plus 2.0pt minus 4.0pt).
%\intextsep : space left on top and bottom of an in-text float (default = 12.0pt plus 2.0pt minus 2.0pt).
% \setlength{\textfloatsep}{4pt}
% \setlength{\intextsep}{4pt}

\usepackage{caption}
% \usepackage[font={small,it}]{caption}
% \usepackage[font={bf}]{caption}
\setlength{\abovecaptionskip}{2pt} % less space between captions and figures
% \setlength{\abovecaptionskip}{-2pt} % less space between captions and figures
% \setlength{\abovecaptionskip}{-5pt}	% less space between captions and figures
% \setlength{\belowcaptionskip}{-13pt}  % less space below captions
% \setlength{\belowcaptionskip}{-10pt}  % less space below captions
\setlength{\belowcaptionskip}{-4pt}	% less space below captions

% \usepackage{paralist}
% \setdefaultleftmargin{10pt}{10pt}{}{}{}{}

% \usepackage{changepage}
% \usepackage{tabulary}

% \usepackage{outlines}
\usepackage{enumitem}
\newcommand{\ItemSpacing}{0mm}
\newcommand{\ParSpacing}{0mm}
% \setenumerate[1]{itemsep={\ItemSpacing},parsep={\ParSpacing},label=\arabic*.}
% \setenumerate[2]{itemsep={\ItemSpacing},parsep={\ParSpacing},label=\arabic*.}
\setenumerate[2]{itemsep={\ItemSpacing},parsep={\ParSpacing}}

% \usepackage[linewidth=1pt]{mdframed}
% \mdfsetup{frametitlealignment=\center, skipabove=0, innertopmargin=1mm,
% innerleftmargin=2mm, leftmargin=0mm, rightmargin=0mm}

%-------------------------------------------------------------- Miscellaneous setup

% \newlength\myindent
% \setlength\myindent{2em}
% \newcommand\bindent{%
%   \begingroup
%   \setlength{\itemindent}{\myindent}
%   \addtolength{\algorithmicindent}{\myindent}
% }
% \newcommand\eindent{\endgroup}

% remove unwanted space between paragraphs;
% it's set by the IEEE conference format, but no papers from this conference have it
% \parskip 0ex plus 0.2ex minus 0.1ex

% make vectors be bold instead of with arrows
\renewcommand{\vec}[1]{\mathbf{#1}}
% add 'mat' command to make matrices bold
\newcommand{\mat}[1]{\mathbf{#1}}

\DeclarePairedDelimiter\ceil{\lceil}{\rceil}
\DeclarePairedDelimiter\floor{\lfloor}{\rfloor}

\newtheorem{Definition}{Definition}[section]

% ------------------------ convenience commands
\newcommand\eps\varepsilon
\DeclareMathOperator{\infimum}{inf}
\renewcommand\inf\infty

\renewcommand{\b}[1]{\textbf{#1}}

\renewcommand{\c}{\vec{c}}
\newcommand{\q}{\vec{q}}
\renewcommand{\r}{\vec{r}}
\newcommand{\x}{\vec{x}}
\newcommand{\xhat}{\hat{\vec{x}}}
\newcommand{\y}{\vec{y}}
\newcommand{\yhat}{\hat{\vec{y}}}
\newcommand{\z}{\vec{z}}
\newcommand{\zhat}{\hat{\vec{z}}}
\newcommand{\X}{\mat{X}}

\newcommand{\R}{\mathbb{R}}
\newcommand{\vtheta}{\vec{\theta}}
\newcommand{\vdelta}{\vec{\delta}}

% \newcommand{\rshift}{\texttt{>>}\text{ }}
% \newcommand{\lshift}{\texttt{<<}\text{ }}
\renewcommand{\rshift}{\texttt{>>}\text{ }}
\renewcommand{\lshift}{\texttt{<<}\text{ }}

\newcommand{\onehalf}{\frac{1}{2}}

\newcommand{\piOver}[1]{\frac{\pi}{#1}}

\renewcommand{\sp}{\text{ }}

% \newcommand{\pitwo}{\frac{\pi}{2}}
% \newcommand{\pifour}{\frac{\pi}{4}}

% \newcommand{\norm}[1]{\left\lVert #1 \right\rVert}
\DeclarePairedDelimiter\abs{\lvert}{\rvert}%
\DeclarePairedDelimiter\norm{\lVert}{\rVert}%

\DeclareMathOperator{\Beta}{Beta}
\DeclareMathOperator{\Normal}{\mathcal{N}}
\DeclareMathOperator{\erf}{erf}
\DeclareMathOperator{\Var}{Var}
\DeclareMathOperator{\sign}{sign}
\DeclareMathOperator{\log2}{\text{log}_2}

% make *all* text 10pt
% \renewcommand{\footnotesize}{\normalsize}
% \renewcommand{\footnotesize}{\small}
% \renewcommand{\small}{\normalsize}


\newcommand{\mine}{\textsc{Sprintz}}

\begin{document}

% \setcopyright{rightsretained}

%Conference
% \copyrightyear{2017}
% \acmYear{2017}
% \setcopyright{acmlicensed}
% \acmConference{KDD '17}{August 13--17, 2017}{Halifax, NS, Canada}\acmPrice{15.00}\acmDOI{10.1145/3097983.3098195}
% \acmConference{KDD'17}{}{August 13--17, 2017, Halifax, NS, Canada.} \acmPrice{15.00} \acmDOI{10.1145/3097983.3098195}
% \acmISBN{978-1-4503-4887-4/17/08}
% \acmISBN{ISBN 978-1-4503-4887-4/17/08}

% \fancyhead{}  % apparently KDD printer wants no headers so can add their own
\vldbTitle{Sprintz: Time Series Compression for the Internet of Things}
\vldbDOI{https://doi.org/TBD}

% ================================================================
\title{Sprintz: Time Series Compression \\ for the Internet of Things}
% ================================================================

% \author{Davis W. Blalock}
% \affiliation{
%   \institution{Computer Science and Artificial \\ Intelligence Laboratory}
%   \institution{Massachusetts Institute of Technology}
% }
% \email{dblalock@mit.edu}

% \author{John V. Guttag}
% \affiliation{
%   \institution{Computer Science and Artificial \\ Intelligence Laboratory}
%   \institution{Massachusetts Institute of Technology}
% }
% \email{guttag@mit.edu}

\numberofauthors{3}

\vldbAuthors{Davis Blalock, Sam Madden, John Guttag}

\author{
    \alignauthor
    Davis Blalock    \\
    \affaddr{Computer Science and Artificial Intelligence Laboratory}\\
    % \affaddr{Massachusetts Institute of Technology} \\
    \affaddr{MIT} \\
    \email{dblalock@mit.edu}
    \alignauthor
    Sam Madden    \\
    \affaddr{Computer Science and Artificial Intelligence Laboratory}\\
    % \affaddr{Massachusetts Institute of Technology} \\
    \affaddr{MIT} \\
    \email{madden@csail.mit.edu}
    \alignauthor
    John Guttag    \\
    \affaddr{Computer Science and Artificial Intelligence Laboratory}\\
    % \affaddr{Massachusetts Institute of Technology} \\
    \affaddr{MIT} \\
    \email{guttag@csail.mit.edu}
}

\maketitle

% ------------------------------------------------
\begin{abstract}
% ------------------------------------------------

% By 2020, mobile and connected devices will be produce over \%d\% of the world's data. Much of

Thanks to the rapid proliferation of connected devices, sensor-generated time series constitute a large and growing portion of the world's data. In order to reduce storage, power, and transmission costs, it is desirable to compress this data to the greatest extent possible. % Many time series databases propose some method of doing this, but, to the best of our knowledge, there is neither a systematic assessment of what approaches are most effective nor a compression algorithm that clearly outperforms others on such data. % Unfortunately, existing compression algorithms either provide only limited compression when applied to time series, degrade data quality, or are not suitable for execution on low-power processors.

We introduce a time series compression algorithm that achieves state-of-the-art compression ratios, high compression speed, and extreme decompression speed. As we show experimentally, storing data using our algorithm not only saves space, but also accelerates many queries of practical interest relative to existing compression schemes. Moreover, because our approach requires $<$1KB of memory, it can be run on low-power sensors at the network edge, offloading computation from the central database and reducing network load.

% We also provide an open-source benchmark library that can be used to assess dozens of different queries and compressors on various datasets.

Extensive experiments on datasets from many domains show that these results hold not only for sensor data, our primary application of interest, but also across a wide array of other time series. % Our approach also outperforms even domain-specific algorithms designed for more powerful hardware in some cases.

\end{abstract}

% ------------------------------------------------
% CCS taxonomy stuff / keywords
% ------------------------------------------------


% \begin{CCSXML}
% <ccs2012>
% <concept>
% <concept_id>10003752.10003809.10010031.10002975</concept_id>
% <concept_desc>Theory of computation~Data compression</concept_desc>
% <concept_significance>500</concept_significance>
% </concept>
% <concept>
% <concept_id>10003752.10003809.10010047</concept_id>
% <concept_desc>Theory of computation~Online algorithms</concept_desc>
% <concept_significance>300</concept_significance>
% </concept>
% <concept>
% <concept_id>10010583.10010588.10003247</concept_id>
% <concept_desc>Hardware~Signal processing systems</concept_desc>
% <concept_significance>100</concept_significance>
% </concept>
% </ccs2012>
% \end{CCSXML}

% \ccsdesc[500]{Theory of computation~Data compression}
% \ccsdesc[300]{Theory of computation~Online algorithms}
% \ccsdesc[100]{Hardware~Signal processing systems}

% \keywords{Compression, Time Series, Performance}

% ================================================================
\section{Introduction} \label{sec:intro}
% ================================================================


% Time series are ubiquitous and only growing in importance thanks to the proliferation of sensor data from autonomous vehicles, smart phones, wearables, and other connected devices. Although a huge volume of this data is stored in modern databases---many of them designed specifically for time series []---there has been relatively little work on how best to compress this data.

% Alternative 1st paragraph:

% Thanks to the exponentially increasing [] number of embedded, mobile, and wearable devices producing sensor data, there is a rapidly growing need to store time series. This has given rise to numerous time series databases to manage and query this data, both in industry [] and academia [].

% In order to store the data efficiently, all of these systems require some form of compression algorithm.

Time series are ubiquitous and only growing in importance thanks to the proliferation of sensor data from autonomous vehicles, smart phones, wearables, and other connected devices. Much of this data is currently stored in databases---many of them designed specifically for time series \cite{respawnDB, openTSDB, chronicleDB, kairosDB, druid, influxDB, gorilla}. Consequently, an improved means of compressing time series could greatly enhance the storage, network, and computation efficiency of many database systems.

% Moreover, a compression algorithm optimized for time series could support many

% there has been relatively little work on how best to compress such data. An improved method of doing so could increase storage, network, and computation efficiency for nearly all databases storing time series.

% Although a huge volume of this data is currently stored in databases---many of them designed specifically for time series \cite{respawnDB, openTSDB, chronicleDB, kairosDB, druid, influxDB, gorilla}---there has been relatively little work on how best to compress such data. An improved method of doing so could increase storage, network, and computation efficiency for nearly all databases storing time series.

 % As a result, existing systems are not only using more storage space than is necessary, but are also % to represent time series at rest.

% Most modern databases store data in a compressed form

% Lack of a  only means increased storage requirements, but also reduced performance.

From a compression perspective, time series have four attributes uncommon in other data.

\begin{enumerate}
    \item \textbf{Lack of exact repeats}. In text or structured records, there are many sequences of bytes---often corresponding to words or phrases---that will exactly repeat many times. This makes dictionary-based methods a natural fit. In time series, however, the presence of noise makes exact repeats less common \cite{extract, epenthesis}.
    \item \textbf{Multiple variables}. Real-world time series often consist of multiple variables that are collected and accessed together. For example, the Inertial Measurement Unit (IMU) in modern smart phones collects three-dimensional acceleration, gyroscope, and magnetometer data, for a total of nine variables sampled at each time step. These variables are also likely to be read together, since each on its own is insufficient to characterize the phone's motion. %These variables are all sampled and fed to the processor together, resulting in every ninth value being from the same variable. This means that there is often a strong lag correlation in the data (in this case with a lag of nine). % , for a total of nine variables sampled at each time step. % These variables are all sampled and fed to the processor together, resulting in every ninth value in memory order being
    \item \textbf{Low bitwidth}. Any data collected by a sensor will be digitized into an integer by an Analog-to-Digital Converter (ADC). Nearly all ADCs have a precision of 32 bits or fewer \cite{digikeyADCs}, and typically 16 or fewer of these bits are useful. For example, even lossless audio codecs store only 16 bits per sample \cite{flac, shorten}. Even data that is not collected from a sensor can often be stored using six or fewer bits without loss of performance for many tasks \cite{epenthesis, mdlIntrinsic, sax}.
    \item \textbf{Temporal correlation}. Since the real world usually evolves slowly relative to the sampling rate, successive samples of a time series tend to have similar values. However, when multiple variables are present and samples are stored contiguously, this correlation is often present only with a lag---e.g., with nine IMU variables, every ninth value is similar. Such lag correlations violate the assumptions of most compressors, which treat adjacent bytes as the most likely to be related.
    % quantization even to a mere six bits [] rarely harms classification accuracy, and quantizing to two sometimes improves it [].
\end{enumerate}

% Existing databases

At present, even databases specialized for time series typically use general-purpose compression algorithms that are not optimized for these characteristics. Common codecs chosen include dictionary-based methods such as LZ4 \cite{lz4} or Snappy \cite{snappy}, floating-point compression algorithms \cite{gorilla}, or generic integer compression algorithms \cite{influxDB, simple8b}, possibly with some form of invertible preprocessing \cite{influxDB, gorilla, berkeleyTreeDB}.

%such as LZ4 \cite{lz4} or Snappy \cite{snappy}, floating-point compression algorithms \cite{gorilla}, or generic integer compression algorithms \cite{influxDB, simple8b}, possibly with some form of invertible preprocessing \cite{influxDB, gorilla, berkeleyTreeDB}.

These methods can be effective, but have several drawbacks. First, they can be performance bottlenecks. Buevich et al. report that ``Enabling compression adds a factor of six slowdown'' \cite{respawnDB} in their reads, even with the relatively fast gzip decoder. Second, as we show experimentally, there is considerable room for improvement in compression ratios. Finally, because of the nature of time series data and workloads, there are a number of requirements for a time series compression algorithm that existing approaches do not satisfy. In particular:

% These methods are effective to some extent, but there is considerable room for improvement. As Buevich et al. report, ``Enabling compression adds a factor of six slowdown.''

% as we show experimentally, there is considerable room for improvement. Moreover, because of the nature of time series data and workloads, there are a number of desirable attributes for a time series compression algorithm that these approaches often lack. In particular:

% Because of the nature of time series data and workloads, an ideal compression algorithm would have the following characteristics, in addition to high compression ratio:

% Unfortunately, each of these approaches has drawbacks. Moreoever, as we show experimentally, it is possible to achieve a much better speed-compression tradeoff than what current methods offer.

% In addition to high compression ratio, it is desirable for a time series compression algorithm to have the following properties:

\begin{enumerate}
\itemsep0mm
% \item \b{Support for fast scans}. Time series workloads are not only read-heavy \cite{respawnDB, berkeleyTreeDB, influxDB}, but often necessitate scans through the data []. This means not only that high decompression speed is essential, but that it is desirable to push down queries directly to the compressed representation.
\item \b{Small block size}. To accelerate queries over arbitrary time intervals, many databases use indexing structures that partition time series into many small segments \cite{berkeleyTreeDB, respawnDB}. A compression algorithm must therefore be effective even on small numbers ($<$1024) of samples. This also enables offloading compression to clients, which may be low-power sensors with only a few kilobytes of RAM \cite{respawnDB}.
\item \b{High decompression speed}. Time series workloads are not only read-heavy \cite{respawnDB, berkeleyTreeDB, influxDB}, but often necessitate materializing data (or downsampled versions thereof) for visualization, clustering, computing correlations, or other operations \cite{respawnDB}. At the same time, writing is often append-only \cite{gorilla, respawnDB}, so compression need only be fast enough to keep up with the rate of data ingestion.
% \item \b{Low memory}. To reduce network usage and offload computation to clients, it is desirable for the compressor to run on clients []. Unfortunately, clients producing time series data are often low-power sensors with only a few KB of RAM [].
% \item \b{High decompression speed}. While compression speed is also desirable, time series workloads are often read-heavy [] or even append-only [], meaning that decompression will run many more times than compression. Moreover, as mentioned above, compression can often be carried out at the client.
% \item \b{Exploit related columns}. Time series often consist of several signals that will almost always be accessed together. For example, one would rarely access only X-axis acceleration without Y-axis and Z-axis, or longitude without latitude. It is desirable to make accessing related columns together inexpensive.
% \item \b{Amenable to low-bitwidth integer data}. Most real-world signals are digitized using an Analog-to-Digital Converter (ADC). This means that the data can be represented as integers of at most 32 bits \cite{digikeyADCs}, and typically 16 or fewer. For example, even lossless audio codecs store only 16 bits \cite{flac, shorten}. % Furthermore, there is empirical evidence that most time series can be quantized to 6 or fewer bits with little or no loss of information []. % (as measured by misclassification rate) []. % Consequently, we focus on compressing 8b and 16b integer time series. % Data coming from sensors will almost always be digitized by an Analog-to-Digital Converter (ADC)
\item \b{Lossless}. Given that time series are almost always noisy and often oversampled, it might not seem necessary to compress them losslessly. However, noise and oversampling 1) tend to vary across applications, and 2) can be addressed in an application-specific way as a preprocessing step. Consequently, instead of assuming that some level of downsampling or some particular smoothing will be appropriate for all data, it is better for the compression algorithm to preserve what it is given and leave preprocessing up to the application developer.
\end{enumerate}

Existing methods used in databases almost universally violate Requirement 1. This is because modern general-purpose compressors tend to require large dictionaries of previously seen byte strings to be effective \cite{lz4, snappy, gzip, zlib}. Examples of such compressors used in databases are LZ4 \cite{lz4, chronicleDB, rocksDB, druid}, Snappy \cite{snappy, openTSDB, influxDB, kairosDB, parquet}, and gzip \cite{respawnDB, parquet}, among others. In the time series literature, existing algorithms either violate Requirement 3 through lossy compression \cite{sax, tsCompressSmartGrid, ecgCompressLossy, apca, lemireSegmentation}, or are specialized for floating-point data \cite{gorilla} or timestamps \cite{gorilla, berkeleyTreeDB, fastpfor}.

% The primary contribution of this work is \minesp (Sprintz PReserves Integer Time Series),
The primary contribution of this work is \mine,
a compression algorithm for integer time series that offers state-of-the-art compression ratios and speed while also satisfying all of the above requirements. It needs $<$1KB of memory, can use blocks of data as small as eight samples, and can decompress at up to 3GB/s in a single thread. \mine's effectiveness stems from exploiting 1) temporal correlations in each variable's value and variance, and 2) the potential for parallelization across different variables, realized through the use of vector instructions.

A key component of its operation is a novel, vectorized forecasting algorithm for integers. This algorithm can simultaneously train online and generate predictions at close to the speed of \texttt{memcpy}, in addition to significantly improving compression ratios compared to delta coding.

The remainder of this paper is structured as follows. In Section~\ref{sec:method}, we introduce relevant definitions and describe our method. In Section~\ref{sec:results}, we evaluate \minesp in comparison to existing algorithms across a number of publicly available datasets. We also discuss when \minesp is most advantageous based on these results. We defer an overview of related work to Section~\ref{sec:relatedWork} in order to first provide context regarding our problem and methodology.

 % and speed, uses $<$1KB of memory, and can use blocks of data as small as eight samples.

% As part of \mine, we also introduce \fire (Fast Integer REgression),
% As part of \mine, we also introduce \fire,
% an online forecasting algorithm for predictive coding. \fire can simultaneously train on and generate predictions for over 5GB of data per second per thread, a rate currently surpassed only by branch predictors and other algorithms with direct hardware support.


% % ================================================================
% \section{Definitions and Problem} \label{sec:problem}
% % ================================================================

% \input{problem.tex}

% ================================================================
\section{Related Work} \label{sec:relatedWork}
% ================================================================


% ------------------------------------------------
\subsection{Compression of time series}
% ------------------------------------------------

Most work on compressing time series has focused on lossy techniques. In the data mining literature, Symbolic Aggregate Approximation (SAX) \cite{SAX} and its variations \cite{isax, isax2} dominate. These approaches preserve enough information about time series to support specific data mining algorithms (e.g. \cite{fastShapelet, hotSax}), but are extremely lossy; a hundred-sample time series might be compressed into one or two bytes, depending on the exact SAX parameters.

Other lossy approaches include Adaptive Piecewise Constant Approximation \cite{apca}, Piecewise Aggregate Approximation \cite{paa}, and numerous other methods \cite{swab, lemireSegmentation, tsCompressSmartGrid} that approximate time series as sequences of low-order polynomials.

For audio time series specifically, there are a large number of lossy codecs \cite{vorbis, shorten, aac, opus}, as well as a small number of lossless \cite{flac, alac} codecs. In principle, some of these could be applied to non-audio time series. However, modern codecs make such strong assumptions about the possible numbers of channels, sampling rates, bit depths, or other characteristics that it is infeasible to use them on non-audio time series.

Many fewer algorithms exist for lossless time series compression. For floating-point time series, the only algorithm of which we are aware is that of the Gorilla database \cite{gorilla}. This method XORs each value with the previous value to obtain a diff, and then bitpacks the diffs. In contrast to our approach, it assumes that time series are univariate and have 64-bit floating-point elements. % The same work also describes a method of compressing integer timestamps. This consists of first delta-delta coding and then applying a similar bitpacking compression approach.
% Like other databases \cite{influxDB, berkeleyTreeDB}, they delta-delta code before applying compression %Most time series databases use some form of integer compression (c.f. next section) [], generic predictive coding \cite{akumuli}, or floating-point compression methods.

For lossless compression of integer time series (including timestamps), existing approaches include directly applying general-purpose compressors \cite{respawnDB, openTSDB, chronicleDB, kairosDB, druid}, (double) delta encoding and then applying an integer compressor \cite{influxDB, gorilla}, or predictive coding and byte-packing \cite{akumuli}. These approaches can work well, but tend to offer both less compression and less speed than \mine.

% A final noteworthy method is Blosc \cite{blosc}. While not intended solely for time series (or integers), Blosc's assumption that every $k$ bytes (or bits \cite{bitshuf}) will be correlated for some $k$ is a natural fit for multivariate time series. %grouping of correlated bytes and/or bits makes it well-suited for multivariate time series.


% ------------------------ parquet
%   -currently supports snappy, GZIP, lzo
%   -hdfs also supports these; prolly others also
%   -https://www.cloudera.com/documentation/enterprise/5-6-x/topics/impala_parquet.html#parquet_compression
% ------------------------ Akumuli
%   -DFCM predictor (for both floats and ints?); XOR with prediction, then do something to pack it
%   -"The timestamp can be a simple integer or datetime in ISO 8601 format"
%   -handles int or float values
%   -"This data-structure consists of 4KB blocks"
%   -"They require large amount of memory per data stream to maintain a sliding window of previously seen samples. The larger the context size the better compression ratio gets."
%   -"if we're dealing with 100'000 time-series we'll need about 1GB of memory only for compression contexts."
% ------------------------ RespawnDB
%   -GZIP
%   -"Enabling compression adds a factor of six slowdown in BTDS performance"
% ------------------------ OpenTSDB
%   -LZ0 or snappy for floats, varbyte (1, 2, 4, or 8 bytes) for ints
%   -http://opentsdb.net/faq.html
% ------------------------ Gorilla
%    -custom F64 for values, custom delta-delta for timestamps
% ------------------------ ChronicleDB
%   -lz4
% ------------------------ BerkeleyTreeDB
%   -custom delta-delta
% ------------------------ KairosDB
%     -off-the-shelf compressors: lzo, snappy, probably others
%     -https://github.com/kairosdb/kairosdb/search?utf8=✓&q=compression&type=
% ------------------------ Druid
%   -"The timestamp and metric columns are simple: behind the scenes each of these is an array of integer or floating point values compressed with LZ4."
%       -"metrics" are scalars (the time series)
%   -they also have "dimensions" which are categorical cols; assign each a numeric ID and map IDs to lists of places they occur
%   -http://druid.io/docs/latest/design/segments.html
% ------------------------ InfluxDB
%   -"Timestamp encoding is adaptive and based on the structure of the timestamps that are encoded. It uses a combination of delta encoding, scaling, and compression using simple8b run-length encoding"
%   -"Floats are encoded using an implementation of the Facebook Gorilla paper."
%   -For integers: "If all ZigZag encoded values are less than (1 << 60) - 1, they are compressed using simple8b encoding. If any values are larger than the maximum then all values are stored uncompressed in the block. If all values are identical, run-length encoding is used."
%   -"Strings are encoding using Snappy compression"
%   -and before compaction (where all the above methods get used): "When a write comes in the new points are serialized, compressed using Snappy, and written to a WAL file"
%       -and note: "This means that batching points together is required to achieve high throughput performance. (Optimal batch size seems to be 5,000-10,000 points per batch for many use cases.)"
% ------------------------ LittleTable
%   -LZO1X (LZO) compression http://www.oberhumer.com/opensource/lzo/
% ------------------------ RocksDB (not a ts database, but whatever)
%   -snappy, lz4, zstd


\subsection{Compression of integers}

% Integer compression is not as well-studied as general-purpose compression, but has seen great progress in recent years.

The fastest methods of compressing integers are generally based on bit packing---i.e., using at most $b$ bits to represent values in $\{0, 2^b-1\}$, and storing these bits contiguously \cite{bbp, pfor, fastpfor}. Since $b$ is determined by the largest value that must be encoded, naively applying this method yields limited compression. To improve it, one can encode fixed-size blocks of data at a time, so that $b$ can be set based on the largest values in a block instead of the whole dataset \cite{kGamma, pfor, fastpfor}. A further improvement is to ignore the largest few values when setting $b$ and store their omitted bits separately \cite{pfor, fastpfor}.

\minesp bit packing differs significantly from existing methods in that it compresses much smaller blocks of samples. This reduces its throughput as compared to, e.g., \cite{fastpfor}, but significantly improves compression ratios, since large values only increase $b$ locally. It also allows significantly lowers the memory requirements.

A common \cite{flac, shorten} alternative to bit-packing is Golomb coding \cite{golomb}, or its special case Rice coding \cite{rice}. The idea is to assume that the values follow a geometric distribution, often with a rate constant fit to the data. %, and therefore make the encoding cost linear in the magnitude of the encoded value.

Both bit packing and Golomb coding are bit-based methods in that they do not guarantee that encoded values will be aligned on byte boundaries. When this is undesirable, one can employ byte-based methods such as 4-Wise Null Suppression \cite{kGamma}, LEB128 \cite{dwarf}, or Varint-G8IU \cite{varintG8IU}. These methods reduce the number of bytes used to store each sample by encoding in a few bits how many bytes are necessary to represent its value, and then encoding only that many bytes. Some, such as Simple8B \cite{simple8b} and SIMD-GroupSimple \cite{groupSimd}, allow fractional bytes to be stored while preserving byte alignment for groups of samples. % These methods allow for efficient universal codes---that is, codes that can represent any possible integer. Universal

% Before applying any of these coding schemes, it is common to apply some transform to the raw data to make the values closer to 0. Such transforms include delta encoding, \cite{fastpfor, bbp}, delta-delta encoding \cite{influxDB}, and linear predictive coding (LPC) \cite{flac}. LPC deterministically generates a prediction for each sample based on the previous samples, and stores the error in the prediction instead of the raw value; when the errors are small, the integers stored are closer to 0. Delta coding and delta-delta coding are special cases wherein each sample is predicted to be the previous sample, or a linear extrapolation from the previous two samples, respectively.


\subsection{General-purpose compression}
While \minesp is not intended to be a general-purpose compression algorithm, a reasonable alternative to using a specialized method would be to apply a general-purpose compression algorithm, possibly after delta coding or other preprocessing. Thanks largely to the development of Asymmetric Numeral Systems (ANS) \cite{ans} for entropy coding, general purpose compressors have advanced greatly in recent years. In particular, Zstd \cite{zstd}, Brotli \cite{brotli}, LZ4 \cite{lz4} and others have attained speed-compression tradeoffs significantly better than traditional methods such as GZIP \cite{gzip}, LZO \cite{lzo}, etc. As described in Section~\ref{sec:intro}, however, these methods have much higher memory requirements that \mine.

% Also of note is Blosc \cite{blosc}, which is especially applicable to multivariate time series as a result of its grouping correlated bits and/or bytes together during preprocessing.

% ------------------------------------------------
\subsection{Predictive Filtering}
% ------------------------------------------------

% TODO move delta, double, LPC descriptions to here. Also talk about \fire.

% Predictive coding in some form is a common element of many compression algorithms. This consists of having some forecaster predict the values of the next byte(s) to be compressed and only storing the prediction error, rather than the original value. When the forecaster is better than chance, the errors will be drawn from a lower-entropy distribution than that of the raw data. In particular, they will often be tightly concentrated around zero \cite{shorten}. The original data can be reconstructed from the errors by having the decoder run the same forecaster and add the encoded errors to its predictions.

For numeric data such as time series, there are four types of predictive coding commonly in use: predictive filtering \cite{png}, delta coding \cite{fastpfor, bbp}, double-delta coding \cite{influxDB, gorilla}, and XOR-based encoding \cite{gorilla}. In predictive filtering, each prediction is a linear combination of a fixed number of recent samples. This can be understood as an Autoregressive model or the application of a Finite Impulse Response (FIR) filter. When the filter is learned online, this is termed ``adaptive filtering.''

Delta coding is a special case of predictive filtering where the prediction is always the previous value. Double-delta coding, also called delta-delta coding or delta-of-deltas coding, consists of applying delta coding twice in succession. XOR-based encoding is similar to delta coding, but replaces subtraction of the previous value with the XOR operation. This modification is often desirable for floating-point data \cite{gorilla}.

\fire can be understood as a special case of adaptive filtering. While adaptive filtering is a well-studied mathematical problem in the signal processing literature, we are unaware of a practical algorithm that attains speed within an order of magnitude of that of \fire. %. or that accounts for the subtleties of low-precision integers.

% Indeed, the only methods

% most similar method is likely that of \cite{neuralBranchPredictor}, in which a

% Moreover, existing databases almost universally use delta, double-delta, or XOR encoding, suggesting that simple  % suggesting that these approaches are the relevant benchmarks.

% Before applying any of these coding schemes, it is common to apply some transform to the raw data to make the values closer to 0. Such transforms include delta encoding, \cite{fastpfor, bbp}, delta-delta encoding \cite{influxDB}, and linear predictive coding (LPC) \cite{flac}. LPC deterministically generates a prediction for each sample based on the previous samples, and stores the error in the prediction instead of the raw value; when the errors are small, the integers stored are closer to 0. Delta coding and delta-delta coding are special cases wherein each sample is predicted to be the previous sample, or a linear extrapolation from the previous two samples, respectively.


% ================================================================
\section{Method} \label{sec:method}
% ================================================================


% We totally have a method. And golly, does it ever method.

\mine is a bit packing-based predictive coder. It consists of four components:
\begin{enumerate}
\itemsep0em
\item \b{Forecasting.} \minesp employs a forecaster to predict each sample based on previous samples. It encodes the difference between the next sample and the predicted sample, is typically closer to zero than the next sample itself.
\item \b{Bit packing.} \minesp then bit packs the errors as a ``payload'' and prepends a header with sufficient information to invert the bit packing.
\item \b{Run-length encoding.} If a block of errors is all zeros, \minesp waits for a block in which some error is nonzero and then writes out the number of all-zero blocks instead of the (otherwise empty) payload.
\item \b{Entropy coding.} \minesp Huffman codes the headers and payloads.
\end{enumerate}

These components are run on blocks of eight samples (motivated in Section~\ref{sec:bitpacking}), and can be modified to yield different compression-speed tradeoffs. Concretely, one can 1) elide entropy coding for greater speed and 2) choose between delta coding and our online learning method as forecasting algorithms. Our method is slightly slower but often improves compression.

Before describing the four components in greater detail, we first provide an outline of the overall algorithms for compressing and decompressing blocks.

% We elaborate upon each of the four components in the following subsections, but first provide an outline of the overall compression and decompression functions.

\newcommand{\err}{\texttt{err}}
\newcommand{\nbits}{\texttt{nbits}}
\newcommand{\packed}{\texttt{packed}}
\newcommand{\buff}{$\texttt{buff}$}
\newcommand{\bytes}{\texttt{bytes}}
% \newcommand{\header}{\texttt{header}}
\newcommand{\payload}{\texttt{payload}}
\newcommand{\f}{\texttt{f}}
\newcommand{\fore}{\texttt{forecaster}}
\newcommand{\self}{\texttt{self}}

% ------------------------ compress overview

An overview of how \minesp compresses one block of samples is shown in Algorithm~\ref{algo:compress}. In lines \ref{line:bodyStart}-\ref{line:encPredictEnd}, \minesp predicts each sample based on the previous sample and any state stored by the forecasting algorithm. For the first sample in a block, the previous sample is the last element of the previous block, or zeros for the initial block. In lines \ref{line:eachColStart}-\ref{line:bodyEnd}, \minesp \\ determines the number of bits required to store the largest error in each column and then bit packs the values in that column using this many bits. ((Recall that each column is one variable of the time series). If all columns required 0 bits, \minesp continues reading in blocks until some error requires $>$0 bits (lines \ref{line:rleLoopStart}-\ref{line:rleLoopEnd}). At this point, it writes out a header of all 0s and then the number of all-zero blocks. Finally, it writes out the number of bits required by each column in the latest block as a header, and the bit packed data as a payload. Both header and payload are compressed with Huffman coding.

\begin{algorithm}[h]
% \caption{encodeBlock($\{\x_1, \ldots, \x_B \}, \vtheta $)}
\caption{encodeBlock($\{\x_1, \ldots, \x_B \}, \fore$)}
\label{algo:compress}
\begin{algorithmic}[1]

\State{Let \buff\sp be a temporary buffer}

\For {$i \leftarrow 1,\ldots,B$} \COMMENTT {For each sample} \label{line:bodyStart}
    % \State{$ \tilde{\x}_i, \vtheta \leftarrow $predictAndTrain$(\x_{i-1}, \vtheta)  $}
    \State{$ \tilde{\x}_i \leftarrow $ $\fore$.predict$(\x_{i-1})$}
    \State{$ \err_i \leftarrow \x_i - \tilde{\x}_i  $}
    \State{$\fore$.train($\x_{i-1}$, $\x_i$, $\err_i$)} \label{line:encPredictEnd}
\EndFor
\For {$j \leftarrow 1,\ldots,D$} \COMMENTT {For each column} \label{line:eachColStart}
    \State{$ \nbits_j \leftarrow \max_i\{ $requiredNumBits$(\err_{ij}) \} $}
    \State{$ \packed_j \leftarrow $ bitPack$(\{x_{1j},\ldots,\x_{Bj} \}) $}  \label{line:bodyEnd}
\EndFor

\LineComment{Run length encode if all errors are zero}
\If{$\nbits_j$ \texttt{==} $0$, $1 \le j \le D$}
    \Repeat  \COMMENT{Scan until end of run} \label{line:rleLoopStart}
        \State{Read in another block and run lines \ref{line:bodyStart}-\ref{line:bodyEnd} }
    \Until {$\exists_j [\nbits_j \neq 0 ]$} \label{line:rleLoopEnd}
    % \State{Read in another block and run lines \ref{line:bodyStart}-\ref{line:bodyEnd} until some nbits$_j$ != $0$}
    \State{Write $D$ $0$s as headers into \buff}
    \State{Write number of all-zero blocks as payload into \buff}
    \State{Output huffmanCode(\buff)}
\EndIf

\State{Write $\nbits_j$, $j = 1,\ldots,D$ as headers into \buff}
\State{Write $\packed_j$, $j = 1,\ldots,D$ as payload into \buff}
\State{Output huffmanCode(\buff)}

% \RETURN {$ odds_{event} - \max(odds_{noise}, odds_{next}) $}
\end{algorithmic}
\end{algorithm}

% ------------------------ decompress

\minesp begins decompression (Algorithm~\ref{algo:decomp}) by decoding the Huffman-coded bitstream into a header and a payload. Once decoded, these two components are easy to separate since the header is first and always of fixed size. If the header is all 0s, the payload indicates the length of a run of zero errors. In this case, \minesp runs the predictor until the corresponding number of samples have been predicted. Since the errors are zero, the forecaster's predictions are the true sample values. In the nonzero case, \minesp unpacks the payload using the number of bits specified for each column by the header.

\begin{algorithm}[h]
% \caption{decodeBlock(\bytes, $B$, $D$, $\vtheta$)}
\caption{decodeBlock(\bytes, $B$, $D$, $\fore$)}
\label{algo:decomp}
\begin{algorithmic}[1]

\State{$\nbits$, $\payload$ $\leftarrow$ huffmanDecode(\bytes, $B$, $D$) }

% \State{$\f \leftarrow \texttt{Forecaster()} $}
\If{$\nbits_j$ \texttt{==} $0$ $\forall j$} \COMMENT{Run length encoded}
    \State{$\texttt{numblocks} \leftarrow $ readRunLength()}
    \For {$i \leftarrow 1,\ldots,(B $ $\cdot$ \texttt{numblocks})}
        % \State{$ \tilde{\x}_i, \vtheta \leftarrow $ $\fore$.predict$(\x_{i-1}, \vtheta) $}
        \State{$ \x_i \leftarrow $ $\fore$.predict$(\x_{i-1})$}
        \State{Output $\x_i$}
        \State{$\fore$.train($\x_{i-1}$, $\x_i$, 0)}
    \EndFor
    % \RETURN {}
% \EndIf
\Else \COMMENT{Not run-length encoded}
% \State{}
\For {$i \leftarrow 1,\ldots,B$}
    % \State{$ \tilde{\x}_i, \vtheta \leftarrow $ $\fore$.predict$(\x_{i-1}, \vtheta)  $}
    \State{$ \tilde{\x}_i \leftarrow $ $\fore$.predict$(\x_{i-1})$}
    \State{$ \err_i \leftarrow $unpackErrorVector$(i$, \nbits, \payload$) $}
    \State{$ \x_i \leftarrow \err_i + \tilde{\x}_i  $}
    \State{Output $\x_i$}
    \State{$\fore$.train($\x_{i-1}$, $\x_i$, $\err_i$)}
\EndFor
\EndIf
\end{algorithmic}
\end{algorithm}

% ------------------------------------------------
\subsection{Forecasting}
% ------------------------------------------------

\begin{figure*}[t]
\begin{center}
    \includegraphics[width=\textwidth]{paper/overview}
    \caption{Overview of \mine\text{ }using a delta coding predictor.\textit{ a)} Delta coding of each column, followed by zigzag encoding of resulting errors. The maximum number of significant (nonzero) bits is computed for each column. \textit{b)} These numbers of bits are stored in a header, and the original data is stored as a (byte-aligned) payload, with leading zeros removed. When there are few columns, each column's data is stored contiguously. When there are many columns, each row is stored contiguously, possibly with padding to ensure alignment on a byte boundary.}
    % With blocks of eight samples (only four shown for clarity), this ensures that each column's data begins on a byte boundary.
    \label{fig:overview}
\end{center}
\end{figure*}

\minesp forecasting can use either delta coding or \fire (Fast Integer REgression), a novel online forecasting algorithm we introduce.

Forecasting with delta coding consists of predicting each sample $\x_i$ to be equal to the previous sample $\x_{i-1}$. This method is stateless given $\x_{i-1}$ and is extremely fast. It is particularly fast when combined with run-length encoding, since it yields a run of zero errors if and only if the data is constant. This means that decompression requires only copying a fixed vector, with no additional forecasting or training. Moreover, when answering queries, one can sometimes avoid decompression entirely []---e.g., one can compute the max of all samples in the run by computing the max of only the first value.

Forecasting with \fire is slightly more expensive but often yields better compression. An overview of \fire prediction and training are given in Algorithm~\ref{algo:xff}. A \fire forecaster is parametrized by three values (line~\ref{line:xffCtor}): the number of columns $D$, the learning rate $\eta$, and the bitwidth of the integers stored in the columns $w$. Internally, the forecaster also maintains an accumulator for each column (line~\ref{line:counter}) and the difference (delta) between the two most recently seen samples (line~\ref{line:deltas}).

To predict, the forecaster first derives a coefficient for each column based on the accumulator. By shifting the accumulator by $\log2(\eta)$ bits, the forecaster obtains a learning rate of $2^{-\log2(\eta)} = \eta$. It then estimates the next delta as the product of this coefficient and the previous delta. It predicts the next sample to be the previous sample plus this estimated delta.

Because all values involved are integers, this multiplication is done using twice the bitwidth $w$ of the data type---e.g., using 16 bits for 8 bit data. The product is then right shifted by an amount equal to the bit width. This has the effect of performing a fixed-point multiplication with step size equal to $2^{-w}$.

The forecaster trains by performing a gradient update on the L1 loss between the true and predicted samples. I.e., given loss:
\begin{align}
    \mathcal{L}(x_i, \tilde{x}_{i}) = \abs{x_{i} - \tilde{x}_{i}}
    = \abs{x - (x_{i-1} + \frac{\alpha}{2^w} \cdot \delta_{i-1})} \\
    = \abs{\delta_{i} - \frac{\alpha}{2^w} \cdot \delta_{i-1}}
\end{align}
for one column's value $x_i = \x_{ij}$ for some $j$ and coefficient $\alpha$, the gradient is:
\begin{align}
    % x = 5
    % \frac{\partial x}{\partial \alpha} x = 5
    % \mathcal{L}(\x_i, \tilde{\x}_i)
        \frac{\partial }{\partial \alpha} \abs{\delta_{i} - \frac{\alpha}{2^w} \cdot \delta_{i-1}}
% \begin{equation*}
&= \begin{cases}
        -{2^{-w}}\vdelta_{i-1} & \x_{i} > \tilde{\x}_{i} \\
        {2^{-w}}\vdelta_{i-1} & \x_{i} \le \tilde{\x}_{i}
\end{cases} \\
&= \sign(\eps) \cdot {2^{-w}}\vdelta_{i-1} \\
&\propto \sign(\eps) \cdot \vdelta_{i-1}
% \end{equation*}
\end{align}
where we define $\eps \triangleq \x_{i} - \tilde{\x}_{i}$ and ignore the $2^{-w}$ as a constant that can be absorbed into the learning rate. In all experiments, we set the learning rate $\eta$ to $\frac{1}{2}$. This value is unlikely to be ideal for any particular dataset, but we found in preliminary experiments that it consistently worked reasonably well on all of the datasets we tried. % and was large enough to avoid zeroing out small gradients.

In practice, \fire differs slightly from the above pseudocode in three ways. First, instead of computing the coefficient for each sample, we compute once at the start of each block. Second, instead of performing a gradient update after each sample, we average the gradients of all samples in each block and then perform one update. Finally, we only compute a gradient for every other sample, since this has little or no effect on the accuracy and slightly improves speed.

% ------------------------ xff pseudocode

\begin{algorithm}[h]
% \begin{struct}[h]
% \label{algo:xff}
% \floatname{algorithm}{Algorithm}
\caption{Class FIRE\_Forecaster} \label{algo:xff}
\begin{algorithmic}[1]

\Function{Init}{$D$, $\eta$, $w$} \label{line:xffCtor}
\State $\self$.learnShift $\leftarrow \lg(\eta)$
\State $\self$.bitWidth $\leftarrow w$ \COMMENT{8-bit or 16-bit}
\State $\self$.accumulators $\leftarrow $ zeros($D$) \label{line:counter}
\State $\self$.deltas $\leftarrow $ zeros($D$) \label{line:deltas}
\EndFunction

\Function{Predict}{$\x_{i-1}$}
\State $\texttt{coeffs} \leftarrow \self$.accumulators \rshift $\self$.learnShift
% \State $\texttt{coeffs} \leftarrow (\texttt{coeffs} \text{ }\rshift 4) \text{ }\lshift 4$
\State $\tilde{\vdelta} \leftarrow$ (\texttt{coeffs} $.*$ $\self$.deltas) $\rshift \self$.bitWidth
\RETURN $\x_{i-1} + \tilde{\vdelta}$
\EndFunction

\Function{Train}{$\x_{i-1}$, $\x_{i}$, $\err_i$}
\State $\texttt{gradients} \leftarrow \sign(\err_i) \sp .* self$.deltas
\State $\self$.counters $\leftarrow \self$.counters + $\texttt{gradients}$
\State $\self$.deltas $\leftarrow \x_{i} - \x_{i-1}$
\EndFunction

% % \State{$\nbits$, $\payload$ $\leftarrow$ huffmanDecode(\bytes, $B$, $D$) }

\end{algorithmic}
\end{algorithm}
% \end{struct}

% \begin{algorithm}[h]
% \caption{XFFpredict($\fore$, $\x_{i-1}$)}
% \label{algo:overview}
% \begin{algorithmic}[1]

% \State{$\alpha \leftarrow \fore$.counter \texttt{>>} $\lg(\eta)$}
% \State{$\vdelta_{i-1} \leftarrow \fore$.prev}
% % , $\vdelta_{i-1}$


% \State{Foo}
% % \State{$\nbits$, $\payload$ $\leftarrow$ huffmanDecode(\bytes, $B$, $D$) }

% \end{algorithmic}
% \end{algorithm}

% ------------------------------------------------
\subsection{Bit Packing} \label{sec:bitpacking}
% ------------------------------------------------

An illustration of \mine's bitpacking is given in Figure~\ref{fig:overview}. The prediction errors from delta coding or \fire are zigzag encoded \cite{zigzag} and then the minimum number of bits required is computed for each column. Zigzag encoding is an invertible transform that interleaves positive and negative integers such that each integer is represented by twice its absolute value, or twice its absolute value minus one for negative integers. % We zigzag encode because it is extremely efficient and facilitates computing the number of bits necesasary.

Given the zigzag encoded errors, the number of bits $w^\prime$ required in each column can be computed as the bitwidth minus the fewest leading zeros in any of that column's errors. E.g., in Figure~\ref{fig:overview}a, the first column's largest encoded value is 16, represented as \texttt{00010000}, which has three leading zeros. This means that we require $w^\prime = 8 - 3 = 5$ bits to store the values in this column. One can find this value by ORing all the values in a column together and then using a built-in function such as GCC's $\texttt{\_\_builtin\_clz}$ to compute the number of leading zeros in a single assembly instruction (c.f. \cite{fastpfor}). This optimization motivates our use of zigzag encoding to make all values nonnegative.

Once the number of bits $w^\prime$ required for each column is known, the zigzag-encoded errors can be bitpacked. First, \minesp writes out a header consisting of $D$ unsigned integers, one for each column, storing the bitwidths. Each integer is stored in $\log2(w)$ bits, where $w$ is the bitwidth of the data. Since there are $w+1$ possible values of $w^\prime$ (including 0), width $w-1$ is treated as a width of $w$ by both the encoder and decoder. E.g., 8-bit data that could only be compressed to 7 bits is both stored and decoded with a bitwidth of 8.

After writing the headers, \minesp takes the appropriate number of low bits from each element and packs them into the payload. When there are few columns, all the bits for a given column are stored contiguously (i.e., column-major order). When there are many columns, the bits for each \textit{sample} are stored contiguously (i.e., row-major order). In the latter case, up to seven bits of padding are added at the end of each row so that all rows begin on a byte boundary. This means that the data for each column begins at a fixed bit offset within each row, facilitating vectorization of the decompressor. The threshold for switching between the two formats is a sample width of $32$ bits (see below).

Because the block begins in row-major order and we seek to reconstruct it the same way, the row-major bit packing case is the more natural. For small numbers of columns, however, the row padding can significantly reduce the compression ratio. Indeed, for univariate 8-bit data, it makes compression ratios greater than 1 impossible. This gives rise to the column-major case; using a block size of eight samples and column-major order, each column's data always falls on a byte boundary without any padding. The downside of this approach is that both encoder and decoder must transpose the block; for up to four 8-bit columns or two 16-bit columns, however, this can be done quickly using SIMD shuffling instructions.\footnote{For recent processors with AVX-512 instructions, one could double these column counts, but we refrain from assuming that these instructions will be available.} This gives rise to the cutoff of 32 bit sample width for switching between the formats.

% ------------------------------------------------
\subsection{Entropy Coding}
% ------------------------------------------------

We entropy code the bit packed representation of each block using Huff0, an off-the-shelf Huffman coder \cite{fse}. Note that this is faster than Huffman coding the original data or the errors since the bit packed block is (usually) shorter than the original data. We do not use Finite-State Entropy \cite{fse} since it is slower than Huffman coding and we never observed a meaningful increase in compression ratio.

% ------------------------------------------------
\subsection{Vectorization}
% ------------------------------------------------

Much of \mine's speed comes from vectorization. For headers, the fixed bitwidths for each field and fixed number of fields allows for packing and unpacking with a mix of vectorized byte shuffles, shifts, and masks. For payloads, delta (de)coding, zigzag (de)coding, and \fire all operate on each column independently, and so naturally vectorize. Because the packed data for all rows is the same length and aligned to a byte boundary (in the high-dimensional case), the decoder can compute the bit offset of each column's data one time and then use this information repeatedly to unpack each row. In the low-dimensional case, all packed data fits in a single vector register which can be shuffled/masked appropriately for each possible number of columns. This is possible since there are at most four columns in this case. On an \texttt{x86} machine, bitpacking and unpacking can be accelerated with the \texttt{pext} and \texttt{pdep} instructions, respectively.


% ================================================================
\section{Experimental Results} \label{sec:results}
% ================================================================


To assess \mine's effectiveness, we implemented both it and comparison algorithms in C++. All of our code and raw results are publicly available on the \mine website.\footnote{https://github.com/dblalock/sprintz} This website also contains experiments on additional datasets, as well as thorough documentation of both our code and experimental setups. All experiments use a single thread on a 2013 Macbook Pro with a 2.6GHz Intel Core i7-4960HQ processor.

All reported timings and throughputs are the best of 5 runs. We use the best, rather than average, since this is standard practice in performance benchmarking.

% ------------------------------------------------
\subsection{Datasets}
% ------------------------------------------------

For assessing accuracy, we use several publicly available datasets:
\begin{itemize}[leftmargin=4mm]
\item \textbf{UCR} \cite{ucrTimeSeries} --- The UCR Time Series Archive is a repository of 85 univariate time series datasets from various domains, commonly used for benchmarking time series algorithms. Because each dataset consists of many time series, we concat the first 100 time series from each dataset to from a single longer time series. This is to allow dictionary-based methods to share information across time series (instead of compressing each in isolation).
% Because the file format is delimited text with labels interspersed with data, we extract ``raw'' data by reading the time series within each dataset into a contiguous array of the appropriate data type We concatenated the first 100 examples from each of the 85 time series datasets in the UCR Time Series Archive \cite{ucrTimeSeries} to form 85 longer time series. Before concatenating, we subtracted off the mean from each example and interpolated one sample between its end and the start of the next example to avoid sudden jumps. This processing makes the datasets in some sense synthetic, but the result is an easy-to-reproduce benchmark incorporating time series from dozens of domains. We report aggregate statistics across these datasets.
\item \textbf{PAMAP} \cite{pamap} --- just PAMAP, not PAMAP2
\item \textbf{MSRC-12} \cite{msrc} --- Some description of this; maybe use this
\item \textbf{WARD} \cite{ward} --- Berkeley Ward Dataset; maybe use this
\item \textbf{ECG} \cite{physiobank} --- Some big ECG dataset from physiobank.
\item \textbf{AMPDs} \cite{ampds} --- The Almanac of Minutely Power Datasets describe electricity, water, and natural gas consumption recorded once per minute for two years at a single home. We treat the data from each of these modalities as one dataset and report aggregate performance across all three.
\end{itemize}

TODO maybe split ampds into 3 different datasets and report on each separately.

For datasets stored as delimited files, we first parsed the data into a contiguous, numeric array and then dumped the bytes as a binary file. For datasets that were not integer-valued, we quantized them such that the largest and smallest values observed corresponded to the largest and smallest values representable with the number of bits tested. Note that this is the worst case scenario for our method since it makes the maximizes the number of bits required to represent the data.

For multivariate datasets, we allowed all methods but our own to operate on the data one variable at a time; i.e., instead of interleaving values for every variable, we store all values for each variable contiguously. This corresponds to allowing them an unlimited buffer size in which to store incoming data before compressing it.

% For multivariate datasets, we concatenated the data from each variable to obtain a univariate time series. As discussed previously, one might be able to obtain better performance by jointly compressing the variables, but we defer this to future work since it both makes direct comparison to existing methods more difficult and complicates the algorithm.


% ------------------------------------------------
\subsection{Comparison Algorithms}
% ------------------------------------------------

\begin{itemize}[leftmargin=4mm]
\item \textbf{SIMD-BP128} \cite{fastpfor} --- The fastest known method of compressing integers.
\item \textbf{FastPFOR} \cite{fastpfor} --- An algorithm similar to SIMD-BP128, but with better compression.
\item \textbf{Simple8b} \cite{simple8b} --- An integer algorithm compression algorithm used by InfluxDB \cite{influxDB}.
\item \textbf{LZO} \cite{lzo} --- A stable and widely-used dictionary compressor employed by KairosDB \cite{kairosDB} and LittleTable \cite{littleTable}.
\item \textbf{Snappy} \cite{snappy} --- A general-purpose compression algorithm developed by Google and used by InfluxDB, KairosDB, OpenTSDB \cite{openTSDB}, RocksDB \cite{rocksDB}, the Hadoop Distributed File System \cite{hdfs} and numerous other projects.
\item \textbf{Zstd} \cite{zstd} --- Facebook's state-of-the-art general purpose compression algorithm. It is based on LZ77 and entropy codes using a mix of Huffman coding and Finite State Entropy (FSE) \cite{fse}. It is available in RocksDB \cite{rocksDB}.
\item \textbf{Brotli} \cite{brotli} --- A recent compression algorithm introduced by Google and now standardized as a web content-encoding type.
\item \textbf{LZ4} \cite{lz4} --- A widely-used general-purpose compression algorithm optimized for speed and based on LZ77. Used by RocksDB and ChronicleDB \cite{chronicleDB}.
% \item \textbf{LZ4-HC} \cite{lz4} --- A variant of LZ4 optimized for compression ratio at the cost of compression speed.
\item \textbf{DEFLATE} [] --- The compression algorithm underlying zlib \cite{zlib} and gzip \cite{gzip}. Used by RespawnDB \cite{respawnDB}, the Parquet columnar storage format \cite{parquet}, and HDFS \cite{hdfs}. We use the zlib implementation.
% \item \textbf{BitShuf} \cite{bitshuf} --- LZ4 with the BitShuffle preprocessor, which groups runs of 0 bits when consecutive values are similar and small.
% \item \textbf{Delta+BitShuf} \cite{gzip} --- Like BitShuf, but applied to the delta-encoded representation of the time series.
% \item \textbf{Delta} --- Simple delta encoding followed by...erm...some kind of bit packing or something.
% \item \textbf{DeltaDelta} --- Delta encoding the delta encoding, as done in some popular time series databases \cite{something, influxDB}.
\end{itemize}

% We also assess the above methods when applied to the delta-encoded representation of the time series, as well as the double-delta-encoded representation. We do not do this for FLAC and FastPFOR since they have similar preprocessing steps built in.

% Note that all of the above except FastPFOR, and possibly FLAC with special configuration, require tens of KB, or even tens of MB, of memory, and therefore are unsuitable for many low-power devices.

% ------------------------------------------------
\subsection{Compression Ratio}
% ------------------------------------------------

In order to rigorously assess the compression performance of both \minesp and existing algorithms, it is desirable to evaluate on a large corpus of time series from heterogeneous domains. The clear choice for such a corpus is the UCR Time Series Archive \cite{ucrTimeSeries}, which is almost universally used in the data mining community for evaluating time series classification and clustering algorithms.

Moreover, in order to avoid biasing the results, it is important to choose an appropriate overall metric. One obvious choice would be to simply measure the total size of all compressed datasets compared to the original size. Unfortunately, this would cause a small number of large datasets to dominate. Even if we took a fixed number of time series per dataset, datasets with longer time series would still contribute most of the data.

Another option would be to compute the compression ratio for each dataset and average these numbers. This is also undesirable for similar reasons. Specifically, it allows performance on a small number of highly compressible datasets to dominate the overall metric.

Because these considerations exactly parallel those associated with comparing classifiers across multiple datasets, we use the methodology outlined in \cite{cdDiagrams}. This means computing the rank of each algorithm for each dataset and comparing the mean ranks using a Nemenyi test. A rank of 1 indicates the best ratio, while 2 indicates the second-best ratio, and so on.

Results using this methodology are shown in Figure~\ref{fig:ratioCD}. Sprintz on high compression settings is significantly better than any existing algorithm. On slightly lower settings, it is still as effective as the best current methods (Zlib and Zstd).

\begin{figure}[h]
\begin{center}
    \includegraphics[width=\linewidth]{paper/cd_diagram_8b_deltas=0}
    \includegraphics[width=\linewidth]{paper/cd_diagram_16b_deltas=0}
    \caption{Compression performance of different algorithms on the UCR Time Series Archive, as measured by mean rank. Lower ranks are better. Methods joined with a horizontal black line are not statistically significantly different.}
    \label{fig:ratioCD}
\end{center}
\end{figure}

As a more intuitive illustration, we also include the distributions of raw compression ratios (Figure~\ref{fig:ratioBox}). \minesp exhibits consistently strong performance across almost all datasets. High-speed codecs such as Snappy, LZ4, and the integer codecs (FastPFOR, SIMDBP128, Simple8B) hardely compress most datasets at all.

Perhaps counter-intuitively, 8-bit data tends to yield higher compression ratios than 16-bit data. This is a product of the fact that the number of bits that are ``predictable'' is roughly constant. I.e., suppose that an algorithm can correctly predict the four most significant bits of a given value; this enables a 2:1 compression ratio in the 8-bit case, but only a 16:12 = 4:3 ratio in the 16-bit case. A more concise explanation is that all but the top few bits are difficult to distinguish from noise, so the larger the fraction of the value these bits constitute, the greater the compressibility.

\begin{figure}[h]
\begin{center}
    \includegraphics[width=\linewidth]{paper/boxplot_ucr_deltas=0}
    \caption{Boxplots of compression performance of different algorithms on the UCR Time Series Archive. Each boxplot captures the distribution of one algorithm across all 85 datasets.}
    \label{fig:ratioBox}
\end{center}
\end{figure}

% Using the 85 UCR datasets:
% \begin{enumerate}
% \item CD Diagram of \mine (all levels) vs other algorithms
% \item CD Diagram of \mine (all levels) vs delta coding + other algorithms
% \item If we don't win in at least the first one, CD Diagram of \mine (all levels) vs delta coding + other algorithms that use very little memory
% \end{enumerate}

% % ------------------------------------------------
% \subsection{Ratio-Speed Tradeoff}
% % ------------------------------------------------

% A natural question is whether the above compression ratios come at the price of reduced speed. To test this, we recorded the speeds and compression ratios of both our method and others on several multivariate datasets. We do not simply reuse the UCR datasets because they are all univariate, which is both not our algorithm's focus and its worst case.


% ------------------------------------------------
\subsection{Decompression Speed}
% ------------------------------------------------

To systematically assess the speed of \minesp, we ran it on datasets with varying numbers of columns and varying levels of compressibility. Specifically, we generated two synthetic datasets: 1) 100 million random values uniformly distributed across the full range of those possible for the given bitwidth; and 2) 100 million random values using only the lower fourth of the bits. The former is incompressible, while the latter allows compression by a factor of two or more. It doesn't allow compression by a factor of four for \minesp since the independent elements make delta coding and \fire counterproductive.

We compressed each of these datasets with \minesp set to treat it as if it had 1 through 80 columns. Numbers that do not evenly divide 100 million result in \minesp memcpy-ing the trailing bytes.

As shown in Figure~\ref{fig:ndims_vs_speed}, \minesp becomes faster as the number of columns increases and as the number of columns approaches multiples of 32 for 8-bit data or 16 for 16-bit data. These values correspond to the 32B width of a SIMD register on the tested machine. These results demonstrate that there is small but consistent overheaded associated with using \fire over delta coding, but that both approaches are extremely fast. Without Huffman coding, \mine decompresses at multiple GB/s except for very few ($<$5) columns.

\begin{figure}[h]
\begin{center}
    \includegraphics[width=\linewidth]{paper/ndims_vs_speed}
    \caption{\minesp becomes faster as the number of columns increases and as the width of each sample approaches multiples of 32B (on a machine with 32B vector registers). Its speed is insensitive to the compression ratio, as illustrated by the speeds in the top row (low compression) roughly equaling those in the bottom row (high compression).}
    \label{fig:ndims_vs_speed}
\end{center}
\end{figure}


% ------------------------------------------------
\subsection{Compression Speed}
% ------------------------------------------------

% Look, a bar graph. Probably 4 rows x 2 cols of subplots, with row corresponding to one dataset and cols corresponding to {8b, 16b}. Each algorithm gets one bar in each subplot. Ideally, run everything a few times and show standard deviations or error bars. Maybe actually have

% \begin{figure}[h]
% \begin{center}
% % \includegraphics[width=\linewidth, trim={0 1cm 0 0},clip]{moose0}
% \includegraphics[width=\linewidth]{encoding_speed}
% \vspace*{-2mm}
% % \caption{Bolt encodes data vectors significantly faster than existing algorithms.}
% \caption{Bolt encodes both data and query vectors significantly faster than similar algorithms.}
% \label{fig:encoding_speeds}
% \end{center}
% \end{figure}

% \vspace{-2mm}


% ------------------------------------------------
\subsection{Quantization Error}
% ------------------------------------------------

% Given that many time series as stored as floating point values, it is natural to wonder what

While floating point values are not the focus of our paper, a straightforward means of generalizing \minesp to floats is to first quantize the floating point data. The downside of doing this is that, because floating point numbers are not uniformly distributed along the real line, such quantization can be lossy. We therefore carried out an experiment to determine how lossy such quantization tends to be on real data.

We did this by quantizing the UCR time series datasets and measuring the noise that this introduces relative to the variance of the data. Specifically, we linearly offset and rescaled the time series in each dataset such that the minimum and maximum values in any time series correspond to $(0, 255)$ for 8-bit quantization or $(0, 65,535)$ for 16-bit quantization. We then obtained the quantized data by applying the floor function to this linearly transformed data.

To measure the error this introduces, we then inverted the linear transformation and computed the mean squared error between the original and this ``reconstructed'' data. The resulting error values for each dataset, normalized by the dataset's variance, are shown in Figure~\ref{fig:quantize_errs}. These normalized values can be understood as signal-to-noise ratio measurements, where the noise is the quantization error. As the figure illustrates, the quantization error is orders of magnitude smaller than the variance for nearly all datasets, and never worse than one order of magnitude even for 8-bit quantization.

\begin{figure}[h]
\begin{center}
    \includegraphics[width=\linewidth]{paper/quantize_errs}
    \caption{Quantizing floating-point time series introduces error that is orders of magnitude smaller than the variance of the data. For every 10 Decibels of signal-to-noise ratio, the variance is 10 times larger than the quantization error. Even with 8 bits, quantization introduces less than 1\% error on all datasets.}
    \label{fig:quantize_errs}
\end{center}
\end{figure}

This of course does not indicate that all time series can be safely quantized. Two counterexamples of which we are aware are timestamps where microsecond or nanosecond resolution matters, and GPS coordinates, where small decimal places may correspond to many meters. However, the above results suggest that quantization is a suitable means of applying \minesp to floating-point data in many applications. This is bolstered by previous work showing that quantization even to a mere six bits [] rarely harms classification accuracy, and quantizing to two or less sometimes improves it [].

% ------------------------------------------------
\subsection{Query acceleration}
% ------------------------------------------------

\begin{enumerate}
\item Sliding Mean on 8b data?
\item Max on 16b data? (Self-driving car acceleration, averaged over two time steps?)
\item Sliding linear classifier? Examples of action (e.g. "climbing stairs") using MSRC-12 or PAMAP? "Sliding on ice" or something in car data?
\end{enumerate}

We can also compare to Sprintz without pushing queries down into the decompress loop (ie, decompress everything first and then query) to show the benefit of our quasi-pushdown; I say ``quasi'' because, except when data gets run-length-encoded, we still do have to decompress it---just not store the decompressed output.

% ------------------------------------------------
\subsection{Other Findings}
% ------------------------------------------------

We encountered a number of counter-intuitive findings regarding what does and does not improve compression ratio. In the interest of facilitating future research in this area, we briefly describe several of them here.

1) Ordering residuals by relative frequency. Absolute value correlates almost perfectly with relative frequency. % Since we'll plot it anyway, point out that residuals usually aren't Laplace distro; more like a power law (in constrast findings in \cite{shorten} for music). Or maybe make that its own bullet.

2) Residuals are heavier-tailed than a Laplace distribution, but less heavy-tailed than a power law (in constrast findings in \cite{shorten} for music).

3) Nearest-neighbor search. Helped but not worth the bit cost to provide the neighbor index; true for blocks of size 8, 16, 32. (delta then nn) is much bettern than (nn then delta). This is especially interesting given how many motif discovery papers use compression as a heuristic for finding repeating patterns; the disconnect is that they don't include the cost of specifying that a discovered pattern is present.



% ================================================================
\section{Conclusion} \label{sec:conclusion}
% ================================================================

Sprintz (Sprintz PReserves INteger Time serieZ) is a compression algorithm. Behold its majesty.

% ================================================================
\begin{appendix}
\section{Additional Experiments} \label{sec:moreResults}
% ================================================================

\input{appendix.tex}

\end{appendix}
% ================================================================
\vspace{-1mm}
% References
% ================================================================

% \nocite{*}

% \IEEEtriggeratref{27}	% trigger column break to make cols even
% \bibliographystyle{ACM-Reference-Format}
\bibliographystyle{abbrv}
\bibliography{doc}

\end{document}
